\documentclass[]{article}
\usepackage{lmodern}
\usepackage{amssymb,amsmath}
\usepackage{ifxetex,ifluatex}
\usepackage{fixltx2e} % provides \textsubscript
\ifnum 0\ifxetex 1\fi\ifluatex 1\fi=0 % if pdftex
  \usepackage[T1]{fontenc}
  \usepackage[utf8]{inputenc}
\else % if luatex or xelatex
  \ifxetex
    \usepackage{mathspec}
  \else
    \usepackage{fontspec}
  \fi
  \defaultfontfeatures{Ligatures=TeX,Scale=MatchLowercase}
\fi
% use upquote if available, for straight quotes in verbatim environments
\IfFileExists{upquote.sty}{\usepackage{upquote}}{}
% use microtype if available
\IfFileExists{microtype.sty}{%
\usepackage{microtype}
\UseMicrotypeSet[protrusion]{basicmath} % disable protrusion for tt fonts
}{}
\usepackage[margin=1in]{geometry}
\usepackage{hyperref}
\hypersetup{unicode=true,
            pdftitle={488 - Energy Predictor Report},
            pdfauthor={Nick Moreno},
            pdfborder={0 0 0},
            breaklinks=true}
\urlstyle{same}  % don't use monospace font for urls
\usepackage{graphicx,grffile}
\makeatletter
\def\maxwidth{\ifdim\Gin@nat@width>\linewidth\linewidth\else\Gin@nat@width\fi}
\def\maxheight{\ifdim\Gin@nat@height>\textheight\textheight\else\Gin@nat@height\fi}
\makeatother
% Scale images if necessary, so that they will not overflow the page
% margins by default, and it is still possible to overwrite the defaults
% using explicit options in \includegraphics[width, height, ...]{}
\setkeys{Gin}{width=\maxwidth,height=\maxheight,keepaspectratio}
\setlength{\emergencystretch}{3em}  % prevent overfull lines
\providecommand{\tightlist}{%
  \setlength{\itemsep}{0pt}\setlength{\parskip}{0pt}}
\setcounter{secnumdepth}{0}
% Redefines (sub)paragraphs to behave more like sections
\ifx\paragraph\undefined\else
\let\oldparagraph\paragraph
\renewcommand{\paragraph}[1]{\oldparagraph{#1}\mbox{}}
\fi
\ifx\subparagraph\undefined\else
\let\oldsubparagraph\subparagraph
\renewcommand{\subparagraph}[1]{\oldsubparagraph{#1}\mbox{}}
\fi

%%% Use protect on footnotes to avoid problems with footnotes in titles
\let\rmarkdownfootnote\footnote%
\def\footnote{\protect\rmarkdownfootnote}

%%% Change title format to be more compact
\usepackage{titling}

% Create subtitle command for use in maketitle
\providecommand{\subtitle}[1]{
  \posttitle{
    \begin{center}\large#1\end{center}
    }
}

\setlength{\droptitle}{-2em}

  \title{488 - Energy Predictor Report}
    \pretitle{\vspace{\droptitle}\centering\huge}
  \posttitle{\par}
    \author{Nick Moreno}
    \preauthor{\centering\large\emph}
  \postauthor{\par}
      \predate{\centering\large\emph}
  \postdate{\par}
    \date{12/10/2019}


\begin{document}
\maketitle

\hypertarget{introduction}{%
\subsubsection{Introduction}\label{introduction}}

For Kaggle's ``ASHRAE - GREAT ENERGY PREDICTOR III'' data competition,
our primary task is to develop a model that predicts energy consumption
of buildings. This competition defines a building's energy consumption
through four sources - electricity, chilled and hot water, and steam -
measured by per-hour building meter readings. As we are interested in
prediction, the plan is to create a generalized additive model to
predict hourly meter readings for each energy type based on historical
building specifications and weather patterns. This model will then be
evaluated using the Root Mean Squared Logarithmic Error.

The dataset we have to train our model consists of 1449 buildings and
their historical meter readings for an entire year on the hour. It
additionally gives data concerning the size and age of each building, as
well as various temperature metrics grouped by 15 building sites. The
goal is to build a model that helps companies and organizations can
choose their next locations or investment opportunities based on areas
that optimize energy saving and cost efficiency.

\hypertarget{methods}{%
\subsubsection{Methods:}\label{methods}}

While we have introduced our main prediction method of choice, our
dataset did not come cleaned. Thus, it's also important to note our
approach to merging our data and dealing with missing values along with
the details of our GAM model. \#\#\#\# Data Cleaning Basically, actual
meter readings came separate from our building and weather information.
In addition, both datasets detailing building and weather information
included a number of missing values for each variable:

\begin{verbatim}
##     site_id building_id primary_use square_feet  year_built floor_count 
##           0           0           0           0         774        1094
\end{verbatim}

\begin{verbatim}
##            site_id          timestamp    air_temperature 
##                  0                  0                 55 
##     cloud_coverage    dew_temperature  precip_depth_1_hr 
##              69173                113              50289 
## sea_level_pressure     wind_direction         wind_speed 
##              10618               6268                304
\end{verbatim}

To deal with this, I decided to fill in the missing values using the
mice() function in R. I utilized the function to create a mean
matching-based predictive model that imputes values for each missing
data point.\\
There are missing values for building year, floor count, and each
weather metric. Because mice uses a collection of predictors in a
dataset, it made more sense to build the values before merging the
dataset, so only weather variables are being used to predict weather
missing values (same for building). So, I used mice independently on
building data and then weather data.\\
It also occured to me that I may want to use specific time metrics when
modeling, so I deconstructed the timestamp to create five new
predictors; hour, day of the week, numerical day, month, and year. I
used lubridate(), a popular R function that simply accesses a timestamp
and exploits specific pieces. Once the missing data was handled, I
merged the data into one training and one test set that we will use for
modeling. It's worth noting that missing values arose again after the
final merge, so I used the same function but, rather than mean matching,
I chose a simply random sample method as it was a small percentage of
our overall data

\hypertarget{eda}{%
\paragraph{EDA}\label{eda}}

I went with a GAM model mainly because of our relatively small amount of
predictor variables. Even with creating five new predictors, we are
still only left with 20 predictors. My thought was that I would be able
to try many different combinations of predictors - all of whom may have
different relationships with meter readings - in a relatively small
amount of time to see what performed the best. To choose which
predictors to use, and the parameters of each method used to transform
the predictor, I needed to take a deeper look into the set of variables
at hand.\\
Off the bat, I knew three of them would not be of any real use -
\emph{timestamp} is taken care of by the new time predictors,
\emph{time.year} only has one value (2016), and creating four different
meter type models renders \emph{meter\_type} useless. For the rest of
the predictors, I always like to run a simple correlation matrix between
the response and all of the predictors. I ran correlations of meter
readings with building and weather data for each meter type:

next were logical time metrics last were collection of weather site\_id
taken care of through weather

\#\#\#Results

\#\#\#Conclusion / Future Work


\end{document}
